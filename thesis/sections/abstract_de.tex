\Abstract
Automatisierungssysteme digitaler Umspannwerke (SAS) nutzen zur Überwachung und Steuerung zunehmend Informations- und Kommunikationstechnologie.
Dies führt zu neuen Herausforderungen in Bezug auf die Informationssicherheit.
Bestehende Normen wie IEC 61850 und IEC 62351 decken jüngste Entwicklungen, einschließlich der attributbasierten Zugriffskontrolle (ABAC) und der attributbasierten Public-Key-Kryptographie (AB-PKC), nicht ausreichend ab.
Daher schlagen wir ein zertifikatsloses, attributbasiertes, servergestütztes Kryptosystem für digitale Umspannwerke vor, das in neu gebaute oder nachgerüstete digitale Umspannwerke integriert wird.
Um ein SAS gegen domänentypische Angriffe zu schützen, setzt unser Ansatz auf eine obligatorische Authentifizierung, Autorisierung und Zugriffskontrolle für die SAS-Kommunikation.
Unser zertifikatsloser, attributbasierter, servergestützter Authentifizierungsansatz bietet algorithmenagnostische kryptographische Protokolle und Dienste, die als Grundlage für andere Cybersicherheitsmechanismen in einem SAS dienen.
Mit unserem Ansatz heben wir die Vorteile von PKC in einem SAS hervor, einschließlich der leichtgewichtigen und sicheren Schlüsselverteilung sowie der Anpassungsfähigkeit im Hinblick auf erfüllte Sicherheitsanforderungen.
Um die Authentizität, Integrität und Nichtabstreitbarkeit der SAS-Kommunikation zu gewährleisten, verwendet unser Ansatz einen authentifizierten Nachrichtenaustausch, der auf digitalen Signaturen basiert.
Um die Vorteile serverbasierter Kryptographie zu unterstreichen, stellen wir zudem ein serverbasiertes AB-PKC-Signaturschema vor.
Zusätzlich zu unserem Authentifizierungsansatz bieten wir einen servergestützten, attributbasierten Autorisierungs- und Zugriffskontrollansatz an, um unautorisierten Zugriff auf SAS-Geräte zu verhindern.
Wir erweitern das Konzept von ABAC durch die Einführung von Echtzeit-Attributen und zeitabhängiger Richtlinienauswertung.
Zudem delegieren SAS-Geräte die rechenintensive Zugriffskontrolle an Durchführungs- und Entscheidungspunkte.
Darüber hinaus bietet unser Ansatz verschiedene Auswertungsstrategien für Zugriffsrichtlinien, um die Recheneffizienz, die Leistungseffizienz und die Speichernutzung für verschiedene Netzwerkverkehrsmuster zu optimieren.
Um unseren Ansatz zu bewerten, führen wir eine theoretische und experimentelle Evaluation durch, die auf einem Ziel-Frage-Metrik-Ansatz basiert.
Die Bewertung umfasst Sicherheits-, Leistungs- und Kompatibilitätsaspekte unseres Ansatzes.
Für die experimentell durchgeführten Analyse stellen wir eine Implementierung des Ansatzes in Form einer Testumgebung bereit.
Auf der Grundlage dieser Implementierung führen wir eine laborgestützte experimentelle Demonstration der Anwendbarkeit mit Umspannwerksequipment unter Verwendung des GOOSE- und SV-Protokolls durch.
Die Ergebnisse der Evaluation zeigen, dass unser Ansatz eine Lösung zur Verbesserung der Kommunikationssicherheit in digitalen Umspannwerk darstellt.
In Übereinstimmung mit den verwandten Arbeiten zeigen die Ergebnisse zudem, dass die strengen Zeitvorgaben für die Kommunikation in einem SAS eine große Herausforderung für die Informationssicherheit darstellen.
