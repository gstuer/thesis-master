\chapter{Conclusion}
\label{ch:conclusion}
%In this last chapter, we conclude the thesis by presenting the potential future work of the proposed approach.
%The potential future work is discussed in \autoref{sec:conclusion:future_work}.
%Furthermore, in \autoref{sec:conclusion:summary}, we present a summary of the contributions and findings of the thesis.
This concluding chapter presents the potential future work of the proposed approach.
The potential future work is discussed in detail in \autoref{sec:conclusion:future_work}.
Furthermore, \autoref{sec:conclusion:summary} provides a summary of the contributions and findings of the thesis.

\section{Future Work}
\label{sec:conclusion:future_work}
% AB-PKC - Pure Cryptography
Our findings indicate that the CASC-SAS approach, its cryptographic approach CASA, and its authorization and access control approach SABAAC serve to enhance the communication security of a SAS.
In the CASC-SAS approach, the enhancement of communication security was achieved by combining cryptographic services provided by CASA with authorization and access control services provided by SABAAC.
Further research could be conducted to determine, whether this enhancement of security is also achievable by employing a purely cryptographic approach.
To answer this question, we propose the design and realization of an AB-PKC approach that satisfies the requirements of the SAS domain.
In contrast to cryptography-dependent but scheme-agnostic ABAC, as we proposed it in CASC-SAS, the AB-PKC approach could allow the accomplishment of additional security objectives, including privacy and anonymity.

% Encryption \& Decryption
In addition to the changes in the approach paradigm, further research might investigate whether the existing CASC-SAS approach could be expanded to encompass encryption and decryption in conjunction with its signing and verification operations.
While confidentiality for power systems via encryption is explicitly non-recommended for time-critical communication in the IEC standards \cite{IEC62351P6}, further research might elucidate how confidentiality can be achieved even in such time-critical systems.

% Hardware Cryptography Accelerators
With regard to the cryptographic services provided by CASA, further studies could be carried out to evaluate the advantages and disadvantages of hardware-based cryptography acceleration.
It is anticipated that the required computation time will decrease, leading to an increase in message throughput through the utilization of hardware accelerators for cryptographic algorithms.
However, factors such as algorithm compatibility, costs per acceleration unit, and computation time consistency may result in a less beneficial influence on the system than currently expected.

% NTP/PTP/ARP Support \& Time Consistency
The evaluation demonstrated that CASC-SAS is capable of securing application protocols of a SAS, as well as multipurpose transport protocols.
However, network time protocols, such as the Network Time Protocol (NTP) and the Precision Time Protocol (PTP), were bypassed by the PEP entities, as the operation of these protocols is susceptible to temporal inconsistencies resulting from authentication and access control.
Further research could investigate how network time protocols could benefit from being processed by a PEP and what additional requirement and constraints have to be satisfied with regard to computation performance and time consistency.
Moreover, lower-layer network management protocols, such as the Address Resolution Protocol (ARP), were bypassed since these protocols provide services not only to SAS devices but also to auxiliary intermediate devices, including network switches and routers.
Future studies could evaluate the feasibility of CASC-SAS to process these network management protocols and mitigate attacks related to them.

% Redundancy Protocols
Furthermore, future studies could investigate the impact of redundancy protocols in time-critical networks on the operation of our approach.
To this end, CASC-SAS could be deployed and assessed in systems utilizing the Parallel Redundancy Protocol (PRP) or Media Redundancy Protocol (MRP).

% Implementation of CASC-SAS in Switch/SDN Controller
To simplify the architectural complexity of CASC-SAS, reduce the overall costs of deployment, and enable processing of the above-mentioned network protocols, we propose the integration of CASC-SAS into network switches as an alternative realization approach.
For this purpose, further research could investigate the potential of realizing CASC-SAS through the use of Software-Defined Networking (SDN) solutions.
This SDN-based CASC-SAS could aggregate the tasks of multiple PEPs by deploying a virtual PEP for each port of a network switch.
Furthermore, distributed SDN controllers might provide the PAP, PSP, PDP, and CAPP services.

% AI for Policy Management
While the proposed PAP entities provide policy management services for human operators, future research could investigate how CASC-SAS might benefit from the utilization of artificial intelligence (AI).
The integration of AI-based intrusion detection could facilitate the creation and modification of security policies that are enforced within a SAS, thereby enabling our approach to mitigate a wider range of cyberattacks in a timelier manner.

% Deployment of CASC-SAS in Time-Critical Non-SAS Enviroments
In addition to the deployment in a SAS, further research is required to evaluate the applicability of CASC-SAS for other time-critical systems.
Therefore, we propose the evaluation of our approach in time-critical systems that have similar requirements as a SAS.
Systems that might potentially benefit from the enhanced communication security provided by our approach include industry 4.0, robotics, avionics, and medical systems.

%intra \& inter SAS communication
\section{Summary}
\label{sec:conclusion:summary}
To address the increasing relevance of cybersecurity for smart grid systems and to overcome the limitations of existing standards like IEC 62351, we proposed CASC-SAS, a novel cryptography and cybersecurity approach for SASs compliant with IEC 61850.
The two attribute-based and server-aided core concepts CASA and SABAAC represent the central parts of the four-layer dual-path CASC-SAS security architecture.

CASA provides protocols, algorithms, and schemes for PKC-based authentication, to safeguard the integrity, authenticity, and non-repudiation of SAS communication.
As a central component of the CASA approach, we proposed an algorithm-agnostic administration and processing platform, that provides key generation, key distribution, key revocation, and server-aided computation services via the proposed APEX protocol to resource-constrained devices of a SAS.
SABAAC enables the administration and enforcement of ABAC policies for time-critical and time-variable environments.
We introduced the concept of time-dependency for attributes and ABAC policies, and discussed methods to manage, distribute, and enforce such expressive and flexible yet computationally expensive access control policies.
With regard to the management and evaluation of access control policies, this thesis proposed a delegated attribute-based authorization protocol responsible for the policy creation, management, storage, and distribution.
With regard to the enforcement of policies, we introduced a delegated attribute-based access control protocol.
To facilitate the enforcement of policies in time-critical systems, we proposed a novel policy enforcement strategy, which combines server-sided precomputation and client-sided caching of access decisions.

We implemented CASC-SAS using the object-oriented programming paradigm and the programming language Java.
The implementation is published open source on GitHub under the European Union Public Licence (EUPL).
To assess the applicability of our approach for SAS environments, we conducted a theoretical and experimental evaluation based on the implementation.
The experiments and evaluation results were published open source alongside the implementation.
Based on formal methods, the evaluation has shown that CASC-SAS is capable of enhancing the communication security of a SAS.
While SAS-typical cyberattacks can be mitigated by employing the approach, we also discussed the changed attack surface, leading to an increased risk of DoS attacks.
The evaluation demonstrated that our approach is a feasible solution for SAS environments not only security-wise but also cost-wise and due to its highly-compatible BITW concept, that allows retrofitting of existing systems.
Moreover, based on testbed-based and lab-based experiments, the evaluation demonstrated the ability of the approach to secure time-critical message exchanges.
Accordingly, CASC-SAS is capable of securing low-latency MMS, GOOSE, and SV message exchanges between the IEDs and MUs of a SAS.
\todo{TODO: GitHub/EUPL reference einfügen}
\todo{TODO: Finish summary -> Add evaluation results}