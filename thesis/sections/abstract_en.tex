\Abstract
Substation automation systems (SAS) increasingly rely on information and communication technology for monitoring and control.
This leads to new challenges with regard to information security.
Existing standards such as IEC 61850 and IEC 62351 do not sufficiently cover recent developments, including attribute-based access control (ABAC) and attribute-based public key cryptography (AB-PKC).
Therefore, we propose a certificateless attribute-based server-aided cryptosystem for SAS, which integrates into the levels and busses of a newly constructed or retrofitted SAS to enhance its communication security.
To protect a SAS against domain-typical adversarial attacks, our approach employs mandatory authentication, authorization, and access control for SAS communication.

Our certificateless attribute-based server-aided authentication approach provides algorithm-agnostic cryptographic protocols and services that serve as a foundation for other cybersecurity mechanisms in a SAS.
With our approach we emphasize the advantages of PKC in a SAS, including lightweight and secure key distribution as well as malleability with regard to satisfied security requirements.
Accordingly, to safeguard the authenticity, integrity, and non-repudiation of SAS communication, our approach uses authenticated message exchanges based on mandatory digital signatures and signature verification.
Furthermore, as we tailored our approach for time-critical communication, it emphasizes the advantages of server-aided cryptography by providing a server-aided AB-PKC signature scheme.
In addition to our authentication approach, we provide a server-aided attribute-based authorization and access control approach to prevent unauthorized access to SAS devices.
We extend the concept of ABAC by introducing real-time attributes and time-dependent policy evaluation.
To take the strict time and resource constraints of a SAS into account, SAS devices delegate the expressive and flexible yet computationally expensive ABAC to policy enforcement and decision points.
Moreover, our approach provides evaluation strategies for different network traffic patterns to optimize the computation efficiency, power efficiency, and memory utilization.

To evaluate our approach, we conducted a theoretical and experimental evaluation based on a goal-question-metric approach.
The evaluation covers security, performance, and compatibility aspects of our approach.
For the experimentally performed evaluations, we provide a testbed implementation of the approach.
Based on the implementation, we conducted a laboratory-based experimental demonstration of applicability using the GOOSE and SV protocol between an intelligent electronic device, a merging unit, and an I/O box.
The results of the evaluation indicate that our approach is a viable solution to enhance the communication security in a newly constructed or retrofitted substation.
The results also indicate, in accordance with the related literature, that the strict time constraints of the low latency communication in a SAS pose a key challenge for information security.
