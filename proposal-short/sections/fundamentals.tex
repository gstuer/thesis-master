\chapter{Fundamentals}
\label{ch:fundamentals}
The purpose of this chapter is to introduce, define, and describe the fundamental terms and concepts of this thesis proposal.
This chapter provides an introduction for access control in \autoref{sec:accesscontrol}.
Moreover, the relevant fundamentals of asymmetric cryptography are discussed in \autoref{sec:fundamentals:cryptography}.

\section{Attribute-Based Access Control (ABAC)}
\label{sec:accesscontrol}
Access control is the process of granting and denying specific requests to logical or physical services and resources \cite{NIST2022}.
Based on the type of service or resource guarded by the access control, two types of access control can be distinguished.
Physical access control supervises access requests to physical facilities.
Logical access control supervises the access to information and information processing services.
Within the scope of the thesis proposal, the term access control will be used to describe logical access control for IT and OT systems.

Logical access control protects objects like data, services, executable applications, or network devices from unauthorized operations \cite{Hu2014}.
An operation is performed by a subject on a specific object.
To protect an object, its owners establish access control policies.
These policies describe which subjects may perform certain operations on a specific object.
The policies are enforced by a logical component referred to as Access Control Mechanism (ACM) \cite{Hu2014}.
The ACM receives the access request from a subject, decides whether the request should be granted or denied, and enforces the decision taken.
The ACM takes the decision based on a framework called access control model.
The access control model defines the functionalities and environment including subjects, objects, and rules for the ACM to take and enforce a decision.

Attribute-Based Access Control (ABAC) is an access control model enabling access decisions based on attributes associated with subjects, objects, actions, and the environment of a system \cite{JTF2020}.
In other words, in ABAC an access request of a subject to perform operations on an object is decided based on assigned attributes of the subject and object, environment conditions, and a set of policies \cite{Hu2014}.
Within the context of ABAC, an attribute is a characteristic containing information in the form of a name-value pair \cite{Hu2014}.
A subject attribute such as identity, clearance, or department describes the characteristics of a person or non-person entity.
An object attribute such as the object classification, type, or owner describes the resource for which the access is requested.
An operation or action attribute describes the function performed on an object by a subject.
The environment conditions or environment attributes describe the context of an access request.
Environment conditions include dynamic characteristics like time of the day, day of the week, and request location of the subject.

A policy represents a rule based on which an access decision is taken for specific attributes \cite{Hu2014}.
As a consequence, a policy can be seen as a relationship between subject, object, environment, and operation attributes describing under which circumstances the ACM grants or denies an access request.

Role-Based Access Control (RBAC) and Identity-Based Access Control (IBAC) represent special cases of ABAC regarding their attributes used \cite{Hu2014}.
An advantage of ABAC compared to other access control models is the higher flexibility regarding multifactor policy expression.
Moreover, ABAC can take access control decisions based on ad-hoc knowledge and knowledge from separate infrastructure.
This is possible due to ABAC taking decisions at request time by evaluating policies instead of static decision-making as found in IBAC and RBAC.

\section{Asymmetric Cryptography (Public-Key Cryptography)}
\label{sec:fundamentals:cryptography}
Cryptography is a scientific discipline concerned with the study of methodologies, algorithms, schemes, and protocols for the encryption and verification of information~\cite{Barker2016,Barker2020,CNSS2022}.
In other words, cryptography provides means to prevent unauthorized access and to enable the verification of information.
The objective of cryptography is to satisfy specific security goals, including the assurance of confidentiality, integrity, authenticity, and non-repudiation.

A cryptographic system or cryptosystem is a set of cryptographic algorithms~\cite{Menezes1996}.
A cryptosystem comprises sets of valid inputs and outputs as well as required cryptographic keys~\cite{Eckert2023}.
Two important principles for the design of cryptosystems were formulated by \citeauthor{Kerckhoffs1883} and \citeauthor{Shannon1949}.
As stated by \citeauthor{Kerckhoffs1883}, the cryptosystem must not require secrecy and must be able to be known by the adversary without inconvenience~\cite{Kerckhoffs1883}.
According to \citeauthor{Shannon1949}, it shall be assumed that the adversary knows the system being used~\cite{Shannon1949}.
The goal of a cryptosystem is to provide specific cryptographic services such as encryption or verification.
Verification describes the process of proving the integrity, authenticity, or non-repudiation of information~\cite{Boneh2023}.
The verification of information is based on a so-called tag or signature created by a signature algorithm.
Encryption describes the process of transforming plain information called plaintext into an unintelligible form called ciphertext to maintain its secrecy~\cite{Barker2016,Boneh2023}.
The inverse process of encryption is referred to as decryption.

Asymmetric cryptography, also referred to as Public-Key Cryptography (PKC), relies on algorithms which use a pair of two related keys for a cryptographic operation and its inverse operation~\cite{Barker2020,CNSS2022,Eckert2023}.
This pair of related keys consists of a private key and a public key.
The private key must be kept secret.
The public key may be shared without consequences for security, as long as its authenticity and integrity is ensured.
Although the two keys are related, the private key cannot be efficiently derived from the public key.

PKC offers the following advantages over symmetric cryptography~\cite{Barker2020,Eckert2023}:
Firstly, PKC does not require a secure channel or secure protocol to exchange keys.
Secondly, the overall number of required keys using PKC is lower.
Moreover, the number of keys scales linear with the number of communication entities.
For example in a network with $n$ entities, $n$ key pairs or $2n$ keys have to be established.
In the same network, pairwise symmetric cryptography would require $n(n-1)/2$ keys.

Nevertheless, symmetric cryptography has advantages in comparison with asymmetric cryptography~\cite{Barker2020}.
Firstly, symmetric-key algorithms are faster than asymmetric-key algorithms.
Secondly, for a given level of security, symmetric cryptographic keys are shorter.
This reduces the memory and bandwidth requirements for key storage and transmission.

\subsection{Certificateless Public-Key Cryptography}
Certificateless Public-Key Cryptography (CL-PKC) can be seen as an intermediate approach between Identity-Based Public-Key Cryptography (ID-PKC) and certificate-based PKC approaches such as Public Key Infrastructure (PKI)~\cite{AlRiyami2003}.
CL-PKC approaches make use of a Trusted Third Party (TTP) called Key Generating Center (KGC) to generate partial private keys based on an entity's identity and a master key~\cite{AlRiyami2003}.
To obtain the private key, the entity combines the partial private key with a secret value.
Consequently, CL-PKC neither suffers from the key escrow problem nor requires a secure communication channel for the key distribution.
To obtain the public key, the entity generates it based on public parameters and the secret value~\cite{AlRiyami2003}.
Similar to ID-PKC, the public key is not derived from the private key and may therefore exist prior to it.
The only restriction is that the public key and the private key must use the same secret value.

\subsection{Attribute-Based Public-Key Cryptography}
Attribute-Based Public-Key Cryptography (AB-PKC) is a generalization of the ID-PKC concept~\cite{Sahai2005,Goyal2006,Hu2023}.
Attribute-Based Encryption (ABE) combines the principles of ABAC with the concept of PKC.
Therefor, attribute-based policies are integrated into cryptographic algorithms in the form of access structures and attributes.
ABE approaches are classified as either Key-Policy ABE (KP-ABE) or Ciphertext-Policy ABE (CP-ABE), depending on whether the access structure is associated with a key or a ciphertext~\cite{Goyal2006,Bethencourt2007,Hu2023}.
In KP-ABE a secret key is able to decrypt a ciphertext if the attributes of the ciphertext satisfy the key-associated access structure.
Consequently, a data owner cannot control who is able to access the data and has to trust a TTP to issue appropriate keys~\cite{Bethencourt2007}.
In CP-ABE A secret key is able to decrypt a ciphertext if the key-associated attributes satisfy the ciphertext's access structure.
Accordingly, each data owner manages the access control policies for its own data, which makes CP-ABE more flexible and scalable than KP-ABE.
Similar to the concept of ABE, Attribute-Based Signatures (ABS) enable the integration of attributes into signing and verification algorithms~\cite{Li2010,Maji2011}.
