\chapter{Related Work}
\label{ch:relatedwork}
In the following section, the related work of the proposed thesis is presented.
The introduced related work serves as a foundation for the proposed approach presented in \autoref{ch:approach}.
Moreover, the similarities, differences, and applications of the related work within the scope of the proposed approach are going to be highlighted.

%%% Ishchenko2018 - Secure Communication of Intelligent Electronic Devices in Digital Substations
An authenticated communication approach for network packets between IEDs and merging units is presented by \citeauthor{Ishchenko2018} \cite{Ishchenko2018}.
They identified the lack of security in existing IEC 61850 substations and ICSs in general as a key weakness.
In order to mitigate this weakness, \citeauthor{Ishchenko2018} present retrofitting of substations as a viable solution
For this purpose, they introduce a system and bump-in-the-wire device called Security Filter as an add-on device between IEDs and Ethernet-based communication busses using the generic object oriented substation event (GOOSE) or sampled values (SV) protocol.
Security Filter appends message authentication code (MAC) tags to outgoing messages of the IEDs and verifies incoming MAC tags.
As a consequence, the communication busses are secured against unauthenticated messages achieving the security goals integrity and authenticity.
Moreover, the Security Filter approach uses a timestamp to avoid replay attacks.

In order to achieve interoperability with legacy communication systems and compatibility with different substation automation systems, the authors introduce a multimode operation design for Security Filter.
The multimode operation design consists of three operation modes.
In filtering mode Security Filter verifies all packets incoming packets, blocks compromised packets after a certain threshold, and tags all outgoing packets.
Moreover, Security Filter alarms the IED about the compromised packets.
In supervisory mode Security Filter tags selected packets based on a specific rate of packets, verifies tagged packets only, and blocks and alarms when the number of compromised packets exceeds the threshold.
Consequently, supervisory mode leads to a reduced computational effort.
The last mode is called advisory mode.
In advisory mode Security Filter selectively tags and verifies packets based on a specific rate of packets but only triggers alarms and does not block packets after the threshold of compromised packets is reached.
Additionally, the operation of Security Filter can be disabled in case of internal errors allowing all packets to pass through.
\citeauthor{Ishchenko2018} showed that Security Filter is able to meet the IEC 61850 performance requirements of GOOSE and SV using a HMAC and GMAC algorithm even on commodity of-the-shelf ARM hardware.

This thesis proposal introduces an approach similar to the Security Filter approach presented by \citeauthor{Ishchenko2018}.
The architecture and security procedures of the proposed approach are inspired by Security Filter.
We propose extending the concept of authenticated communication within substations by using payload encryption to achieve confidentiality as well as authenticity, integrity, and availability.
Moreover, the proposed approach aims to extend the employed access control from identity-based to attribute-based authorization.
As a consequence, more complex access control policies can be established within a substation or ICS in general.

\todo{PRIMARY: ABASS-SS Sec Papers BWSyncShare: Describe all 8 as a related work foundation for further papers!}
\todo{SECONDARY: Review ABASS-ABAC Materials BWSyncShare papers and select fitting papers for rel. work}
\todo{PAPER: ABAC or RBAC in Substation Context}
\todo{PAPER: Encryption in Substation Context}
\todo{PAPER: Signing and Encrpytion in IOT or ICS + Signcryption with low resources}
\todo{PAPER: Proxy Based Crypto and Authentication -> Reduce load on devices}
\todo{PAPER: Server-Aided Verification (SAV) -> Reduce load on devices}
