\chapter{Related Work}
\label{ch:relatedwork}
In the following section, the related work of the proposed thesis is presented.
The introduced related work serves as a foundation for the proposed approach presented in \autoref{ch:approach}.
Moreover, the similarities, differences, and applications of the related work within the scope of the proposed approach are going to be highlighted.
\todo{Add introduction of subsections to rel work introduction}
\todo{Alternative division: ICT-based security layer/measures (e.g. enc/auth of messages) \& Domain-based security layer/measures (e.g. intrusion detection \& access control)}

\section{Secure Communication in Substations}
%%% Ishchenko2018 - Secure Communication of Intelligent Electronic Devices in Digital Substations
An authenticated communication approach for network packets between IEDs and merging units is presented by \citeauthor{Ishchenko2018} \cite{Ishchenko2018}.
They identified the lack of security in existing IEC 61850 substations and ICSs in general as a key weakness.
In order to mitigate this weakness, \citeauthor{Ishchenko2018} present retrofitting of substations as a viable solution
For this purpose, they introduce a system and bump-in-the-wire device called Security Filter as an add-on device between IEDs and Ethernet-based communication busses using the generic object oriented substation event (GOOSE) or sampled values (SV) protocol.
Security Filter appends message authentication code (MAC) tags to outgoing messages of the IEDs and verifies incoming MAC tags.
As a consequence, the communication busses are secured against unauthenticated messages achieving the security goals integrity and authenticity.
Moreover, the Security Filter approach uses a timestamp to avoid replay attacks.

In order to achieve interoperability with legacy communication systems and compatibility with different substation automation systems, the authors introduce a multimode operation design for Security Filter.
The multimode operation design consists of three operation modes.
In filtering mode Security Filter verifies all packets incoming packets, blocks compromised packets after a certain threshold, and tags all outgoing packets.
Moreover, Security Filter alarms the IED about the compromised packets.
In supervisory mode Security Filter tags selected packets based on a specific rate of packets, verifies tagged packets only, and blocks and alarms when the number of compromised packets exceeds the threshold.
Consequently, supervisory mode leads to a reduced computational effort.
The last mode is called advisory mode.
In advisory mode Security Filter selectively tags and verifies packets based on a specific rate of packets but only triggers alarms and does not block packets after the threshold of compromised packets is reached.
Additionally, the operation of Security Filter can be disabled in case of internal errors allowing all packets to pass through.
\citeauthor{Ishchenko2018} showed that Security Filter is able to meet the IEC 61850 performance requirements of GOOSE and SV using a HMAC and GMAC algorithm even on commodity of-the-shelf ARM hardware.

This thesis proposal introduces an approach similar to the Security Filter approach presented by \citeauthor{Ishchenko2018}.
The architecture and security procedures of the proposed approach are inspired by Security Filter.
We propose extending the concept of authenticated communication within substations by using payload encryption to achieve confidentiality as well as authenticity, integrity, and availability.
Moreover, the proposed approach aims to extend the employed access control from identity-based to attribute-based authorization.
As a consequence, more complex access control policies can be established within a substation or ICS in general.

%%% Rodriguez2021 - A Fixed-Latency Architecture to Secure GOOSE and Sampled Value Messages in Substation Systems
An authentication and encryption approach for substation communication using the protocols GOOSE and SV is presented by \citeauthor{Rodriguez2021} \cite{Rodriguez2021}.
The authors state that GOOSE and SV messages are sensitive to not only availability and integrity but also confidentiality threats.
Therefore, the authors present a hardware architecture for the encryption and authentication of GOOSE and SV packets at wire-speed conforming to IEC 62351:2020.
The hardware architecture consists of six sections for packet processing that can be implemented using FPGAs.
According to \citeauthor{Rodriguez2021}, the architecture design follows three main guidelines to face challenges within substations.
Firstly, the architecture has to be modular to support future revisions of standards, algorithms, and protocols.
Secondly, the architecture has to have high performance by making use of techniques like parallelization and pipelining.
Lastly, the implementation in substation systems must be viable with regard to required area usage and computing power.
The authors conducted the evaluation of the presented architecture using simulation-based and hardware-based timing results.
As stated by the authors, the hardware implementation is able to process GOOSE and SV packets with a fixed latency in the order of microseconds.
Consequently, the authors state that the presented hardware architecture is able to provide integrity and confidentiality without exceeding the maximum delivery time of three milliseconds introduced by IEC 61850 for GOOSE and SV packets.

Besides securing the intra-substation communication based on the GOOSE and SV protocol, the thesis proposal extends the idea of providing integrity, authenticity, and confidentiality to inter-substation and remote communication.
In order to achieve flexibility and interoperability with regard to different ICS environments including different protocols and algorithms used, the proposed approach is software-based instead of hardware-based.
Moreover, the proposed approach does not rely on a symmetric-key algorithm but rather asymmetric-key algorithms.
This is possible due to an increase in processing performance of IT and OT devices nowadays.

%%% Hong2019 - Cyber Attack Resilient Distance Protection and Circuit Breaker Control for Digital Substations
According to \citeauthor{Hong2019} \cite{Hong2019}, new technologies in substations lead to benefits including enhanced reliability, interoperability, and reduced engineering effort and costs.
Besides the benefits, new technologies introduce vulnerabilities that may result in security breaches.
As an example, the authors mention unauthorized remote access to substations through misconfigured security devices, such as firewalls.
Moreover, the authors state that an adversary might not only intrude the substation from outside but also from the inside.
From inside the substation, an adversary may inject false measurements into the process bus or gain access to the station bus to inject forged control signals or change the configuration of devices like IEDs.
In order to protect substations against attacks, \citeauthor{Hong2019} present a domain-based collaborative mitigation approach.
According to the authors, the goal of the approach is to enable substation devices to collaboratively defend against attacks.
For this purpose, the authors propose a distributed security domain layer.
The proposed approach can be employed independently or can complement existing information and communication technology (ICT) security approaches.
As stated by the authors, ICT-based security approaches such as firewalls and intrusion detection systems rely exclusively on ICT domain knowledge, whereas the proposed approach relies on knowledge of the power system domain.
As a consequence, new types of attacks as well as errors caused by substation operators can be detected and mitigated.
\citeauthor{Hong2019} presented three attack scenarios that can be mitigated using the presented domain-based collaborative approach.
The presented attack scenarios are an accidental or malicious IED configuration change, false sensor data injection, and false device command injection.
Collaborating devices can block these attacks by validating sensor data and configuration changes based on measurements and metrics as well as predicting consequences of control actions.

The approach presented in the thesis proposal is inspired by the usage of domain-specific knowledge to detect and block attacks.
The proposed approach uses available domain-specific knowledge to design and implement a substation-specific security system.
Moreover, the incremental framework of the proposed approach for the system design, threat analysis, and mitigation strategy design is based on the research framework presented by \citeauthor{Hong2019}.

\section{Access Control in Substations}
%%% Ruland2018 - Firewall for Attribute-Based Access Control in Smart Grids (10.1109/SEGE.2018.8499306)
An access control approach driven by ABAC policies for smart grid systems including substations is presented by \citeauthor{Ruland2018} \cite{Ruland2018}.
As stated by the authors, communication security enables information confidentiality and integrity but does not protect against internal attacks.
As a consequence, the authors present an access control approach to protect devices from unauthorized access.
The presented access control approach is realized in the form of an access control firewall.
The presented approach is based on an architecture that implements a split station bus.
The split station bus serves the purpose of controlling access requests from devices of the outer bus to devices connected to the inner bus.
The access control firewall connects the outer and inner station bus by processing access requests of connected devices.
On the one hand, within the scope of substations, devices connected to the outer station bus include Human Machine Interfaces (HMI), station computers, and WAN gateways.
On the other hand, the inner station bus connects IEDs and enables low-latency GOOSE or GSSE communication between them.
The access control firewall enforces access request decisions based on ABAC policies.
The ABAC policies used in the presented approach are defined, communicated, and evaluated using the eXtensible Access Control Markup Language (XACML) standard \cite{Oasis2013}.
According to \citeauthor{Ruland2018}, the access request decisions are made by a Policy Decision Point (PDP) that can either be part of the access control firewall or be implemented as an external server on the outer station bus.

The approach presented in the thesis proposal employs ABAC similarly to the access control approach presented by \citeauthor{Ruland2018}.
Besides employing ABAC to secure the communication between devices on the station bus, the proposed approach controls access requests to any device within the substation that requires access control. 
For this purpose, not a single but rather distributed ABAC firewall is used.
As a consequence, the firewall does not represent a communication bottleneck or single point of failure of an ICS in the proposed approach.

%%% Burmester2013 - T-ABAC: An attribute-based access control model for real-time availability in highly dynamic systems (10.1109/ISCC.2013.6754936)
A real-time capable ABAC approach is presented by \citeauthor{Burmester2013} \cite{Burmester2013}.
The presented approach identifies the requirements of cyber-physical systems including confidentiality, integrity, and availability.
In particular, according to the authors, employing ABAC in real-time availability scenarios can be challenging due to the dynamic and large event space determining the attribute values.
In other words, resources may not be available in time leading to events threatening the system state not being addressed within strict time limits.
For this purpose, the authors propose an extended ABAC model that is based on real-time attributes to support availability within the strict time constraints of cyber-physical systems.
A real-time attribute represents an attribute whose value is time-dependent.
The availability of a time-dependent attribute can be expressed with an availability label that is dynamically determined based on user and system events as well as the context of the requested service.
An availability label is referred to as priority if it is associated to a subject attribute, congestion for an object attribute, and criticality for an environment attribute.
The authors demonstrate the real-time ABAC approach for IP multicast in Trusted Computing (TC) compliant networks.
Therefor, the authors propose a congestion control algorithm based on the real-time availability labels.
The proposed algorithm guarantees that high priority packets are delivered timely.
In case of a congestion, lower priority packets may be buffered or dropped to support the real-time requirement of high priority packets.
As stated by the authors, the extended ABAC model is applicable to substation automation systems and medical cyber-physical systems.
\todo{Describe similarities, differences, and applications of the paper and the proposed approach.}

%%% Lee2015 - Role-based access control for substation automation systems using XACML (10.1016/j.is.2015.01.007)
An IEC 61850 and IEC 62351 compliant RBAC approach for substations is presented by \citeauthor{Lee2015} \cite{Lee2015}.
According to the authors, data collection and analysis are key drivers in smart grids leading to an increased requirement for data security and access control of substation devices.
In order to address requirements such as confidentiality and integrity, the authors propose an RBAC approach based on IEC 62351 using XACML \cite{Oasis2013}.
As stated by the authors, the communication within substations can either be classified as session-based TCP/IP client-server communication or Ethernet-based publisher-subscriber communication.
The presented approach focuses on session-based access control for TCP/IP communication on the station bus of substations.
As a consequence, the presented RBAC approach can be employed to process MMS communication between IEDs and devices at station level.
The main contribution of \citeauthor{Lee2015} is an implementation of the presented RBAC approach.
The presented implementation relies on a role-based client-server architecture.
The architecture consists of two interconnected client-server pairs, namely an IEC 61850 client and server as well as a RBAC client and server.
The IEC client sends a request including the client's role to the corresponding IEC server.
The IEC server responds to permitted IEC client requests.
Moreover, the IEC server acts as a Policy Enforcement Point (PEP) by delegating requests to an RBAC client.
The RBAC client transforms an IEC request into an XACML request and sends it to the RBAC server for an access request decision.
The RBAC server serves the purpose of making access request decisions by evaluating access control policies.
An IED of a substation incorporates an IEC 61850 server and RBAC client.
The implementation demonstrates the feasibility of RBAC for substations as prescribed by IEC 62351.
Furthermore, as stated by the authors, the presented implementation is capable of processing and responding to MMS requests within the 500 millisecond time requirement for type 3 messages (low speed messages) prescribed by IEC 61850-5.

Instead of exclusively relying on roles, the approach presented in the thesis proposal employs ABAC to enable the usage of fine-grained and flexible attribute-based access policies.
Moreover, the goal of the proposed approach is to secure any communication within substations including type 1 messages (fast messages) and type 2 messages (medium speed messages) as described by IEC 61850-5.

\todo{10.1016/j.jnca.2015.05.023}
\todo{10.1109/PES.2006.1709324}
\todo{10.1109/CIT.2006.63}
\todo{10.14236/ewic/icscsr19.17}


\todo{PRIMARY: ABASS-SS Sec Papers BWSyncShare: Describe all 8 as a related work foundation for further papers!}
\todo{SECONDARY: Review ABASS-ABAC Materials BWSyncShare papers and select fitting papers for rel. work}
\todo{PAPER: ABAC or RBAC in Substation Context}
\todo{PAPER: Encryption in Substation Context}
\todo{PAPER: Signing and Encrpytion in IOT or ICS + Signcryption with low resources}
\todo{PAPER: Proxy Based Crypto and Authentication -> Reduce load on devices}
\todo{PAPER: Server-Aided Verification (SAV) -> Reduce load on devices}
