\chapter{Approach}
\label{ch:approach}
In the following section, we introduce our proposed security approach for substation automation systems.
With the aim of securing the time-critical communication between resource-constrained devices in a time-variable environment, we propose a security framework for \textbf{S}erver-aided \textbf{A}BAC with \textbf{Z}ero-RTT \textbf{E}ncryption (SAZE).
The SAZE ("Sassy") security framework is able to prevent and mitigate cyberattacks by providing security mechanisms and enforcing mandatory communication policies.

The introduction and discussion of the proposed SAZE approach is organized as follows.
At the beginning of this chapter in \autoref{sec:approach:system_model}, we discuss the field of application of the proposed approach by introducing a system model and defining its requirements.
Based on the presented system model and requirements, we introduce the security model, main concepts, and architecture of the SAZE security framework in \autoref{sec:approach:security_model}.
Finally, we present the proposed evaluation strategies and metrics of the security framework in \autoref{sec:approach:evaluation}.
%\todo{Own idea, concept, protocol, evaluation, proof, testing, expected benefits and limitations}
%\todo{SECURITY: Security Goals (CIA + Privacy Preserving)}
%\todo{IDEA: Wrap packages of arbitrary protocols by encripting payload of ethernet package, and adding a new header with ABAC-SS information. Maybe as HW Middleware? If other endpoint does not support the ABAC-SS wrapping fall back to proofing endpoint via identity provider only.}
%\todo{PERF EVALUATION: 1. Run approach on network simulator, and 2. Run approach on raspberry pi's representing different roles like server, DER, user \dots}

\section{System Model}
\label{sec:approach:system_model}
\todo{Introduction: Substation Automation System}
\todo{Architecture: Devices (Resource-Constrained) \& Layers \& Busses inspired by IEC 61850}
\todo{Communication: Busses \& Message Types (+ Link to External/Internal Requests -> Latency of Transmission) inspired by IEC 61850}
\todo{System Threats + Adversaries + Attack Trees}
\todo{Requirements: Security, Performance, Interoperability, Safety, Availability -> Link to different Busses/Devices}

\section{SAZE Security Model \& Framework}
\label{sec:approach:security_model}
\todo{Architecture: SAZE Components \& Layers \& Busses}
\todo{Security Policies -> Link to Requirements \& Attacks}
\todo{Server-Aided ABAC: is Delegated AC, Evaluation Strategy (With PDP, Ad-Hoc, Time Variable Attributes), Policies (Rule/Pol Types:RT/Static, DSL, appropriate message types for pol. types), Requests(External, Internal), Protocols, Session-Based via OAT (Estab. channel before comm.), Semi-Delegated Authentication, Delegated Authorization}
\todo{Zero-RTT Encryption: Extension of AC, Signcryption, TCS/UCS\dots}
\todo{Realization: HW (BITW) or Software Solution}

\section{Evaluation}
\label{sec:approach:evaluation}
\todo{Evaluation Areas \& Metrics: Econ., Perf., Sec.}
\todo{Network Testbed}
\todo{Network Simulation}

\section{Limitations}
\label{sec:approach:limitations}
\todo{Limitations: No intrusion detection but prevention/mitigation, might not be usable for fast messages, not every crypto for every message type}
