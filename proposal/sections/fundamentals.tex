\chapter{Fundamentals}
\label{ch:fundamentals}
The purpose of this chapter is to introduce, define, and describe the fundamental terms and concepts of this thesis proposal.
Moreover, this chapter provides an introduction into the foundational literature.
The terms and concepts defined within this chapter are assumed to be known in the following chapters. 

At the beginning of this chapter in \autoref{sec:itot} and \autoref{sec:ics} the concepts of information technology, operational technology, and industrial control systems are introduced.
Moreover, this chapter defines the terms information security in \autoref{sec:security} and safety in \autoref{sec:safety} for the scope of the proposed thesis.
Furthermore, this chapter provides an introduction for access control including the two general types of access control, namely discretionary and mandatory access control.
The introduction of access control can be found in \autoref{sec:accesscontrol}.
At the end of this chapter in \autoref{sec:securitypolicymodels} an overview of four different classes of security policy models is presented.
\todo{Fundamentals, e.g Access Control (Role, Attribute \dots), Substations (IEDs, IEC61850, GOOSE/MMS/OpenVPN/Ethernet-Based/Analog \dots)}

\section{Information Technology (IT) \& Operational Technology (OT)}
\label{sec:itot}
The term information technology (IT) encompasses the technological concepts and systems required to create, process, store, present, and communicate information.
In the scope of IT, information is an abstract concept which is represented by so-called data or data objects \cite{Eckert2023}.
The meaning of data originates from or is assigned to a data object by using a specific information interpretation rule.
Data objects can be distinguished based on their abilities by being either passive or active.
Passive data objects can only represent information for storage, whereas active data objects can store and process information.

As stated by \citeauthor{Eckert2023} \cite{Eckert2023} an IT system is a dynamical technical system which is able to process and store information.
An IT system is a part of a sociotechnical system and provides information-based services to more abstract social, economical or political structures.
Moreover, the users of an IT system may have different goals, levels of experience, and technical know-how.

When shifting the scope from abstract information storage and processing to the interaction with the physical world, the term operational technology (OT) arises.
As a consequence, OT describes the application and interactions of information storage and processing procedures in a physical environment.
According to \citeauthor{Stouffer2023} \cite{Stouffer2023} OT encompasses systems and devices interacting with the physical environment directly or through managed devices.
The systems and devices interact with the physical environment by detecting changes through monitoring or by causing changes through control of devices or processes.
In the context of OT systems the term process refers to the part of a system producing an output, whereas a controller represents a part of a system that maintains the conformance with specifications.
Besides the industrial control systems (ICS) further discussed in \autoref{sec:ics}, other examples of OT systems are building automation systems and transportation systems.

Although the evolution from analog systems to OT systems by inserting IT into existing physical systems might provide new functionality and enhance system parameters like costs or performance, new challenges may arise \cite{Stouffer2023}.
Especially the typical long life cycle of OT systems and their unique requirements regarding performance, reliability, security, safety, privacy, and environmental impact have to be taken into account when designing, operating, and maintaining OT systems.
In the following the proposed thesis especially focuses on the security implications as well as the design and implementation of secure OT systems.

\section{Industrial Control Systems (ICS)}
\label{sec:ics}
The term industrial control system (ICS) encompasses different types of control systems consisting of monitoring, control and network components acting together to achieve an industrial objective \cite{Stouffer2015}.
In the scope of the thesis proposal, an ICS represents a specific type of OT system that gathers, processes, and stores information while interacting with a physical environment to achieve an industrial objective.
According to \citeauthor{Stouffer2015} \cite{Stouffer2015} the control in an ICS can be partially or fully automated.
Moreover, an ICS can be configured to operate in three different modes:
\begin{enumerate}
    \item Manual Mode: The ICS is completely controlled by humans.
    \item Open-Loop Control Mode: The output of the system process is controlled by established settings rather than process feedback.
    \item Closed-Loop Control Mode: The ICS uses the process output as feedback to achieve the control objective.
\end{enumerate}

\subsection{Architectures}
ICS as well as generic control systems consisting of multiple interconnected components can be classified regarding their control system architecture.
According to \citeauthor{Galloway2013} \cite{Galloway2013} an ICS architecture, or architecture of an ICS network respectively, is typically deeper regarding the levels of hierarchy than a company network.
Moreover, the technologies including devices as well as the communication links and protocols in an ICS network are often heterogeneous.

In the following sections the two main types of control system architectures are presented.
While these architectures introduce different and partially incompatible concepts, the approaches can be complementing when used on different levels of hierarchy of a complex ICS network.

\subsubsection{Supervisory Control \& Data Acquisition (SCADA)}
Supervisor control and data acquisition (SCADA) is a type of control system architecture.
As stated by \citeauthor{bailey2003} \cite{bailey2003} SCADA refers to a combination of telemetry and data acquisition.
The objective of SCADA is to collect data of a remote process, transfer it to a central site, process and analyze the data, and present it to a human operator via human machine interfaces (HMI).
Moreover, SCADA enables sending of control actions back to the remote process.

The collection of data from devices of a remote process and the delivery of control actions back to the remote devices requires a communication path between the central and remote site \cite{Stouffer2023}.
In the scope of OT and ICS the central site is referred to as control center and the remote sites are referred to a field or field sites.
Specialized network components at the field sites enable remote devices to communicate with the control center via telecommunication technologies.
These specialized network components for ICS networks are further discussed in \autoref{sec:networkcomponents}.
Examples for telecommunication technologies used for the communication are wide area networks (WAN), satellite, cellular, and radio technology.

Although the SCADA approach not necessarily requires a communication network to exist but rather works via direct connection between remote devices and the central site, modern SCADA systems rely on bus-based field networks or Ethernet-based solutions \cite{bailey2003}.
As a consequence, according to \citeauthor{bailey2003} the benefits of modern SCADA approaches are minimal required wiring, plug-and-play installation and replacement of devices, remote access to data from anywhere, easier large-scale data storage, and higher flexibility for visualization and incorporation of real data simulations.
The disadvantages of modern SCADA approaches are the higher complexity of components, the functional limitations induced by the network components, the requirement of better trained employees, the higher reliance on communication networks, and the high prices of intelligent field equipment.

\citeauthor{Stouffer2023} \cite{Stouffer2023} further described four basic communication topologies for modern SCADA networks that were initially introduced by the \citeauthor{aga2006} \cite{aga2006}.
The four topologies introduced are point-to-point, series, series-star, and multi-drop.
The point-to-point topology connects each field device using an individual communication channel.
The series, series-star, and multi-drop topologies use daisy-chaining and switching to connect multiple devices using a single shared channel.
The sharing of a single channel among multiple devices increases the efficiency and operation complexity but decreases the costs and system complexity.

\subsubsection{Distributed Control System (DCS)}

\subsection{Network Components}
\label{sec:networkcomponents}
\todo{Add Galloway2013 difference Company vs. ICS networks}
\todo{Add NIST 800-82r3 difference between IT OT System Security}
\todo{PLC, IED, RTU, Data Concentrator \dots}

\subsection{Substations}
\todo{Introduction: SS is OT, Definition \& Modelling via IEC61850, Availability \& Integrity over Confidentiality, typical communication in a scada system, commonly used protocols, possible attack vectors}

\section{Security}
\label{sec:security}
\todo{aka. Informationssicherheit!}
\todo{Add NIST 800-82r3 difference between IT OT System Security}
\subsection{Objective \& Goals}
\todo{Describe Security Objective as in SecStam, Describe different importance w/in different domains (IT vs OT)}
\subsubsection{Confidentiality}
\subsubsection{Integrity}
\subsubsection{Availability}
\subsubsection{Authenticity}
\subsubsection{Non-Repudiation}
\subsubsection{Privacy}
\subsubsection{Anonymization}
\subsubsection{Pseudonymization}
\todo{see Eckert \& $https://www.ucl.ac.uk/data-protection/guidance-staff-students-and-researchers/practical-data-protection-guidance-notices/anonymisation-and$}

\subsection{Levels \& Categories}
\todo{see SecStammm -> Include security label to "label" entities according to their sec. level}

\subsection{Policy}
\todo{see SecStammm}

\section{Safety}
\label{sec:safety}
\todo{aka. Funktionssicherheit!}

\section{Access Control}
\label{sec:accesscontrol}
\todo{General introduction what AC is exactly. Including what is a subject, object, policy, label, maybe how the concept (-> "Reference Monitor") looks like.}
\todo{Explain Identification/Authorization/Authentication, see Access Control at $https://en.wikipedia.org/wiki/Information_security$}

\subsection{Discretionary Access Control (DAC)}

\subsection{Mandatory Access Control (MAC)}

\section{Classes of Security Policy Models}
\label{sec:securitypolicymodels}

\subsection{Access Control Models}
\subsubsection{Identity-Based Access Control (IBAC)}
\subsubsection{Role-Based Access Control (RBAC)}
\subsubsection{Attribute-Based Access Control (IBAC)}

\subsection{Information Flow Models}
\todo{Introduction + Chinese Wall Model}

\subsection{Multi-Level Security Models}
\todo{Introduction + Bell-La Padula Model}

\subsection{Non-Interference Models}
