\chapter{Fundamentals}
\label{ch:fundamentals}
\todo{Fundamentals, e.g Access Control (Role, Attribute \dots), Substations (IEDs, IEC61850, GOOSE/MMS/OpenVPN/Ethernet-Based/Analog \dots)}

\section{Security}
\label{sec:security}
\todo{aka. Informationssicherheit!}
\subsection{Security Goals}
\todo{Describe different importance w/in different domains (IT vs OT)}
\subsubsection{Confidentiality}
\subsubsection{Integrity}
\subsubsection{Availability}
\subsubsection{Authenticity}
\subsubsection{Non-Repudiation}
\subsubsection{Privacy}
\subsubsection{Anonymization}
\subsubsection{Pseudonymization}
\todo{see Eckert \& https://www.ucl.ac.uk/data-protection/guidance-staff-students-and-researchers/practical-data-protection-guidance-notices/anonymisation-and}

\section{Safety}
\label{sec:safety}
\todo{aka. Funktionssicherheit!}

\section{Access Control}
\label{sec:accesscontrol}
\todo{General introduction what AC is exactly. Including what is a subject, object, policy, label, maybe how the concept (-> "Reference Monitor") looks like.}
\todo{Explain Identification/Authorization/Authentication, see Access Control at https://en.wikipedia.org/wiki/Information_security}

\subsection{Discretionary Access Control (DAC)}

\subsection{Mandatory Access Control (MAC)}

\section{Classes of Security Policy Models}

\subsection{Access Control Models}
\subsubsection{Identity-Based Access Control (IBAC)}
\subsubsection{Role-Based Access Control (RBAC)}
\subsubsection{Attribute-Based Access Control (IBAC)}

\subsection{Information Flow Models}
\todo{Introduction + Chinese Wall Model}

\subsection{Multi-Level Security Models}
\todo{Introduction + Bell-La Padula Model}

\subsection{Non-Interference Models}
