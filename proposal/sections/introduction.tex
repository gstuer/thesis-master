%% LaTeX2e class for student theses
%% sections/content.tex
%% 
%% Karlsruhe Institute of Technology
%% Institute for Program Structures and Data Organization
%% Chair for Software Design and Quality (SDQ)
%%
%% Dr.-Ing. Erik Burger
%% burger@kit.edu
%%
%% Version 1.5, 2024-02-12

\chapter{Introduction}
\label{ch:introduction}
Modern operational technology (OT) such as industrial control systems (ICS) increasingly rely on information and communication technology (ICT) for monitoring and control.
This development leads to new possibilities including the integration of distributed OT into supervisory control and data acquisition (SCADA) systems.
Nevertheless, besides these possibilities, new challenges arise from the increased usage of ICT in OT systems.

According to \citeauthor{Stouffer2023} \cite{Stouffer2023}, the typical long life cycle of OT systems and their unique requirements regarding performance, reliability, security, safety, privacy, and environmental impact have to be taken into account when designing, operating, and maintaining OT systems.
In the following, we focus on the information security of OT systems.
Although a variety of information security solutions exist for information technology (IT), migration of existing approaches to the OT domain may not be a viable solution due to the differing system characteristics, risks, and priorities.
An example for the differing priorities is the information confidentiality.
While the prevention of unauthorized access represents the core objective of IT security approaches, OT systems and especially OT-based critical infrastructure prioritize system availability and reliability.

Within the scope of this thesis, we focus on time-critical communication in a specific type of ICS.
The field of application of the approach proposed by this thesis are so-called substation automation systems (SAS).
A SAS represents the entirety of communication and control equipment of a substation \cite{Padilla2015}.
The IEC 61850 series provides standards for the communication networks of digital energy systems \cite{IEC61850P5}.
The goal of the IEC 61850 series is seamless communication and interoperability of systems in a smart energy grid.
Although standards for the communication in a SAS are provided by the IEC 61850 standards, information security is not an objective of the standards.
In order to overcome this problem, the IEC 62351 standard series was created by the \citeauthor{IEC62351P6}.
Part 6 of the IEC 62351 series provides standardized security means for communication compliant to IEC 61850 \cite{IEC62351P6}.
Moreover, Part 8 of the IEC 62351 series provides a role-based access control concept for power system \cite{IEC62351P8}.

Despite the existence of standards for communication and information security, there are remaining challenges in order to secure SAS communication.
According to \citeauthor{Ishchenko2018} \cite{Ishchenko2018} as well as \citeauthor{Elbez2019} \cite{Elbez2019}, the strict time constraints of the low latency communication in substations are key factors for the information security.
According to the authors, asymmetric cryptography formerly prescribed by the IEC 62351 standards is not appropriate due to computational complexity and latency.

\section{Contribution}
\label{sec:contribution}
With the aim of providing means to enhance the information security in an SAS, we propose a \textbf{C}ertificateless \textbf{A}ttribute-Based \textbf{S}erver-Aided \textbf{C}ryptosystem for \textbf{S}ubstation \textbf{A}utomation \textbf{S}ystems (CASC-SAS).
The main objective of the proposed approach is to provide secure asymmetric cryptographic protocols, algorithms, and schemes for SAS communication.
The provided protocols, algorithms, and schemes aim to satisfy SAS security requirements such as integrity, authenticity, and non-repudiation, while taking the strict time constraints of the SAS domain into account.
Moreover, our cryptosystem comprises a server-aided attribute-based access control (ABAC) approach.
This ABAC approach enables the prevention of unauthorized access.
Our proposed ABAC approach uses the provided cryptographic protocols, algorithms, and schemes as a foundation.
Furthermore, our server-aided ABAC approach aims to demonstrate the applicability of expressive and flexible but yet computationally expensive access control policies in an environment with strict time and resource constraints.

\section{Research Questions}
\label{sec:research_questions}
While standards regarding the communication networks of smart grid systems are widely accepted and used, information security still has to face yet unresolved challenges.
Our proposed approach aims to not only enhance the security of SAS communication by satisfying security objectives, but also takes the specific characteristics, risks, and priorities of OT, ICS, and SAS into account.
Based on this proposed approach, the following research questions are going to be answered in the course of this thesis:
\begin{description}
    \item[RQ1] How can a secure certificateless server-aided public key cryptography approach be designed and implemented, that is able to serve as a foundation for a malleable and extendable cryptosystem in the time-critical SAS environment?\\
    \item[RQ2] How can expressive and flexible but yet computationally expensive access control approaches such as ABAC be employed to enable prevention of unauthorized access in a time-critical SAS environment?\\
    \item[RQ3] How can authentication, authorization, and access control be integrated into a certificateless attribute-based server-aided cryptosystem for time-critical SAS communication?
\end{description}
